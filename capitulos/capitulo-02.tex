\chapter{FUNDAMENTAÇÃO TEÓRICA}\label{cap:revisao}

Esta seção apresenta conceitos e fundamentos teóricos para o desenvolvimento deste trabalho. São apresentados tópicos relacionados a comunicação por satélites, evolução do sistema ARGOS, e técnicas de modulação, codificação e sincronização envolvidas no padrão \gls{PTT-A3}. O objetivo é fornecer uma base de conhecimento para a compreensão dos requisitos técnicos e operacionais do sistema de comunicação proposto.

\section{COLETA DE DADOS SBCDA VIA SATÉLITE }\label{sec:sbcda}

A comunicação por satélites desempenha um papel fundamental na coleta e disseminação de dados ambientais em escala regional e global. No contexto brasileiro, essa função é desempenhada pelo \gls{SBCDA}, operado pelo \gls{INPE}. O \gls{SBCDA} é composto pelos satélites \gls{SCD-1}, \gls{SCD-2} e \gls{CBERS-1} até \gls{CBERS-4} apresentados na \autoref{fig:satelites} abaixo, que orbitam a aproximadamente 750 km de altitude, recebendo informações transmitidas por \gls{PCD} espalhadas pelo território nacional \cite{lima_parallel_2021}.


\begin{figure}[H]
    \centering
    \caption{Satélites para coleta de dados ambientais}
    \label{fig:satelites}
    \begin{minipage}[t]{0.48\linewidth}
      \centering
      \includegraphics[width=0.7\linewidth]{assets/cap2/sat-SCD-1.png}
      \vspace{0.5em}
      \caption*{Satélite SCD-1}
      \small
      \begin{tabularx}{\linewidth}{>{\centering\arraybackslash}X >{\centering\arraybackslash}X}
        \toprule
        \textbf{Parâmetro} & \textbf{Valor} \\
        \midrule
        Massa & 115 kg \\
        Potência Elétrica & 110 W \\
        Vida útil & 4 anos \\
        Altitude média & $\approx$ 750 km \\
        Inclinação orbital & 25$^\circ$ \\
        Período orbital & 99,7 min \\
        \bottomrule
      \end{tabularx}
    \end{minipage}
    \hfill
    \begin{minipage}[t]{0.48\linewidth}
      \centering
      \includegraphics[width=0.7\linewidth]{assets/cap2/sat-CBERS-4.png}
      \vspace{0.5em}
      \caption*{Satélite CBERS-4}
      \small
      \begin{tabularx}{\linewidth}{>{\centering\arraybackslash}X >{\centering\arraybackslash}X}
        \toprule
        \textbf{Parâmetro} & \textbf{Valor} \\
        \midrule
        Massa & 1 980 kg \\
        Potência Elétrica & 2 300 W \\
        Vida útil & 3 anos \\
        Altitude média & $\approx$ 778 km \\
        Inclinação orbital & 98,54$^\circ$ \\
        Período orbital & 100,32 min \\
        \bottomrule
      \end{tabularx}
    \end{minipage}
    \vspace{0.5em}
    
\end{figure}

Esses satélites recebem sinais transmitidos pelas \gls{PCD} na faixa de frequência UHF (401,62 a 401,65 \gls{mhz}) e os retransmitem para as \gls{ETR} localizadas em solo, nas faixas de Banda-S (2267,5 \gls{mhz}). Como operam em órbitas baixas, esses satélites realizam aproximadamente 14 revoluções por dia sobre o território nacional, o que permite ampla cobertura espacial.

Apesar da ampla cobertura, a comunicação com satélites de órbita baixa apresenta um grande desafio técnico, sendo a necessidade de visada simultânea entre a \gls{PCD} transmissora e o satélite, o que limita a janela de transmissão e impõe restrições na coleta contínua de dados. Além disso, o movimento relativo entre a \gls{PCD} e o satélite provoca o chamado efeito Doppler, responsável por deslocamentos na frequência do sinal recebido, podendo atingir até ±79,4 \gls{khz}. Esse desvio precisa ser compensado para garantir a correta demodulação do sinal \cite{rae2005detector, rodrigues_demodulador_2018}.

A confiabilidade do enlace também é impactada por fatores como atenuação no espaço livre, ruídos térmicos, e variações atmosféricas. Para compensar esses fatores, são necessárias técnicas específicas de modulação, sincronização, codificação de dados e planejamento de enlace, de modo a garantir a confiabilidade das mensagens transmitidas.

\subsection{Constelação Catarina}

A Constelação Catarina é um projeto nacional baseado no uso de nanossatélites em órbita baixa, para atuar como um novo braço operacional do \gls{SBCDA}. A Constelação Catarina é composta por pequenos satélites integrados com \gls{SDR}, capazes de receber sinais transmitidos pelas \gls{PCD} no padrão \gls{ARGOS-II}, com planos futuros de migração para o padrão \gls{ARGOS-III} \cite{gomes_otimizacao_2024}.

Diferentemente dos satélites tradicionais, os nanosatélites da Constelação Catarina são projetados para realizar a decodificação e o armazenamento dos dados a bordo, o que permite superar a limitação de visada simultânea entre satélite e \gls{ETR}, ampliando a cobertura do sistema \cite{rodrigues_demodulador_2018}.

A arquitetura dos satélites que compõem a constelação é baseada na integração do transceptor \gls{AD9361} com uma \gls{FPGA} da \gls{Zynq-7000}, formando uma plataforma de \gls{SDR} altamente flexível \footnote{https://www.argos-system.org/wp-content/uploads/2023/01/ARTIC-Chipset-AnSem-Info-sheet.pdf}. Essa configuração permite a reconfiguração remota do hardware, o que é especialmente importante para futuras atualizações de protocolo ou migração para novos padrões de comunicação, como o \gls{PTT-A3}.


\section{EVOLUÇÃO DO SISTEMA ARGOS}\label{sec:quadros}

O \gls{ARGOS-II}, base do \gls{SBCDA} desde 1993, utiliza transmissores do tipo \gls{PTT-A2}, baseados em modulação analógica \gls{PM} com codificação Manchester. Essa versão se mostrou eficiente por muitos anos, mas suas limitações logo se tornaram evidentes, especialmente no que diz respeito à robustez frente a ruído, à largura de banda ocupada e à necessidade de visada simultânea entre \gls{PCD} e satélite para a \gls{ETR} \cite{cnes_services_and_message_formats_ed2_rev2_2006}.

A evolução desse sistema levou ao desenvolvimento do \gls{ARGOS-III}, que introduziu novas técnicas digitais de comunicação. Essa nova geração incorporou transmissores do tipo \gls{PTT-A3} e \gls{PTT-ZE}, os quais se destacam pela adoção de modulação \gls{QPSK}, codificação convolucional e embaralhamento de dados, resultando em maior confiabilidade na transmissão e maior eficiência espectral. Além disso, o \gls{ARGOS-III} permite o armazenamento e retransmissão de mensagens a bordo do satélite para a \gls{ETR} \cite{lima_parallel_2021, rodrigues_demodulador_2018}.

\section{ESPECIFICAÇÕES DO PADRÃO PTT-A3}

O transmissor do tipo \gls{PTT-A3} é um dos formatos definidos na terceira geração do sistema \gls{ARGOS}, projetado para oferecer maior robustez na transmissão e maior eficiência na utilização do espectro de frequência. 

A estrutura de um quadro \gls{PTT-A3} é composta por três campos principais, sendo eles: portadora pura, palavra de sincronismo (preâmbulo) e datagrama. Na \autoref{fig:frame}, a estrutura é apresentada de forma detalhada, considerando que a taxa de transmissão \gls{Rb} é de 400 \gls{bps} \textcite{cnes_services_and_message_formats_ed2_rev2_2006}.

\begin{figure}[H]
	\centering
	\caption{Estrutura do quadro de transmissão ARGOS-3}\label{fig:frame}
	\includegraphics[width=\linewidth]{assets/cap2/frame.pdf}
\end{figure}

\subsection{Portadora pura}

A sequência de transmissão do quadro inicia-se com a portadora contínua ou pura, com duração de 82 ± 2 ms. Durante essa etapa a portadora não transmite dados modulados e é utilizada pelo receptor apenas para realizar a detecção do sinal, bem como para facilitar o processo de sincronização de frequência e fase da portadora. 

A \autoref{fig:portadora_pura_freq} apresenta o sinal da portadora pura no espectro em comparação com o sinal modulado. Nota-se que quando apenas a portadora pura é transmitida, o espectro do sinal é concentrado em uma única frequência, sem componentes laterais. Já o sinal modulado apresenta componentes laterais que se estendem ao redor da frequência da portadora \gls{fc}, formando uma banda de uso do espectro mais ampla \cite{cnes_services_and_message_formats_ed2_rev2_2006}.


\begin{figure}[H]
	\centering
	\caption{Comparação de portadora pura e sinal modulado}\label{fig:portadora_pura_freq}
	\includegraphics[width=\linewidth]{assets/cap2/transmitter_modulator_portadora.pdf}
\end{figure}

O processo de detecção do sinal realizado pelo receptor monitora a presença de sinal que ultrapassa um determinado limiar (\gls{Pt}), dessa forma é fundamental que no receptor o sinal esteja o mais concentrado e com a maior \gls{SNR} possível no momento da detecção, para que a frequência da portadora, \gls{fc}, seja identificada corretamente \cite{cnes_services_and_message_formats_ed2_rev2_2006}.


\subsection{Palavra de sincronismo (preâmbulo)}\label{sec:preambulo}

Logo após a portadora pura, é transmitida uma palavra de sincronismo de 30 bits (correspondente a 15 símbolos \gls{QPSK}), sendo fundamental para identificação do início da mensagem codificada, possibilitando a sincronização para decisão. Essa sequência é conhecida e fixa entre transmissor e receptor, no caso do \gls{PTT-A3} sendo $S = \text{2BEEEEBF}_{16}$, o que permite alinhar corretamente a decisão e identificar o início do bloco de dados úteis. \cite{cnes_services_and_message_formats_ed2_rev2_2006}

A sequência \gls{Sn} é separada em dois vetores distintos, \gls{SIn} e \gls{SQn}, por meio de uma intercalação simples de seus bits. O processo de intercalação consiste em distribuir os bits de forma alternada entre os canais \gls{SIn} e \gls{SQn}, resultando em duas sequências de 15 bits cada, que serão transmitida como preâmbulo \cite{cnes_services_and_message_formats_ed2_rev2_2006}, usando a sequência padrão do \gls{ARGOS-III} podemos representar como

\vspace{-1em}
\begin{equation}
\begin{aligned}
    S_I[n] &= [S_0,\ S_2,\ S_4,\ \dots,\ S_{28}] &\mapsto&  S_I[n] = [1111,\ 1111,\ 1111,\ 111]     \\
    S_Q[n] &= [S_1,\ S_3,\ S_5,\ \dots,\ S_{29}] &\mapsto&  S_Q[n] = [0011,\ 0101,\ 0100,\ 111]
\end{aligned}
\label{eq:intercalacao}
\end{equation}

\noindent Importante destacar que esta palavra não é codificada convolucionalmente ou embaralhada, sendo adicionada ao início do vetor de bits de cada canal após esses blocos.  \cite{cnes_services_and_message_formats_ed2_rev2_2006}. 

\subsection{Datagrama}\label{sec:datagrama}

Após o envio da palavra de sincronismo, inicia-se a transmissão dos dados modulados. Esses dados são organizados segundo a estrutura definida pelo datagrama do padrão \gls{ARGOS-III}, que contém os campos responsáveis pela identificação da plataforma, carga útil de dados e controle de finalização, conforme apresentado na \autoref{fig:datagram} \cite{cnes_services_and_message_formats_ed2_rev2_2006}.

\begin{figure}[H]
	\centering
	\caption{Estrutura do datagrama ARGOS-3}\label{fig:datagram}
	\includegraphics[width=\linewidth]{assets/cap2/datagram.pdf}
    
\end{figure}

\section{ESTRUTURA DE UM DATAGRAMA ARGOS-3}

O datagrama transmitido no padrão \gls{ARGOS-III} possui uma estrutura bem definida, composta por campos de dados do usuário, que carregam as informações provenientes dos sensores da \gls{PCD}, e por campos de cabeçalho, responsáveis por identificar a estação transmissora e informar o comprimento total da mensagem. Esses campos incluem o identificador da PCD, o número de blocos de dados e um bit de paridade, que auxilia na verificação de integridade da informação. Essa organização permite que o sistema receptor interprete corretamente o conteúdo transmitido e associe os dados recebidos à plataforma correspondente.


\subsection{Dados de aplicação}

A primeira etapa na montagem do datagrama consiste na coleta dos dados de aplicação, isto é, os dados que efetivamente contêm informação dos sensores a serem transmitidos da \gls{PCD} para o satélite.

\subsubsection{Sensores}

Cada sensor presente nas \gls{PCD} gera um valor de oito bits correspondente à variável monitorada, possibilitando assim 256 ($2^8$) níveis de monitoramento para cada sensor. A PCD pode ser equipada com diferentes sensores, de acordo com o cenário de instalação e os parâmetros ambientais de interesse. 

Entre os sensores comumente utilizados, destacam-se os de direção do vento ($^{\circ}$), precipitação (mm), pressão atmosférica (mB), radiação solar acumulada (MJ/m2), temperatura do ar ($^{\circ}\text{C}$), umidade relativa ($\%$), e velocidade do vento (m/s). Por exemplo, os dados \footnote{http://sinda.crn.inpe.br/PCD/SITE/novo/site/historico/action.php} coletados da PCD 31855 são mostrados no \autoref{quadro:dados-meteorologicos}.


\begin{quadro}[H]
\caption{Dados meteorológicos da PCD 31855 (10/10/2007 - 11/10/2007)}\label{quadro:dados-meteorologicos}
\resizebox{\textwidth}{!}{%
\begin{tabular}{lccccccc}
    \toprule
    \textbf{DataHora} & \textbf{DirVento} & \textbf{Precip.} & \textbf{PressãoAtm} & \textbf{RadSolAcum} & \textbf{TempAr} & \textbf{UmidRel} & \textbf{VelVento} \\
    \midrule
    2007-11-10 21h & 0 & 0 & 945.5 & 2.3  & 30.8 & 25.6 & 0 \\
    2007-11-10 18h & 0 & 0 & 943.8 & 8.75 & 36.5 & 20.8 & 0 \\
    2007-11-10 15h & 0 & 0 & 947.3 & 9.72 & 33.6 & 28.8 & 0 \\
    2007-11-10 12h & 0 & 0 & 950.1 & 4.98 & 28.0 & 51.2 & 0 \\
    2007-11-10 09h & 0 & 0 & 949.3 & 0.17 & 20.9 & 60.8 & 0 \\
    2007-11-10 06h & 0 & 0 & 947.8 & 0.00 & 21.8 & 52.8 & 0 \\
    2007-11-10 03h & 0 & 0 & 948.0 & 0.00 & 23.5 & 40.0 & 0 \\
    2007-11-10 00h & 0 & 0 & 948.4 & 0.00 & 21.9 & 35.2 & 0 \\
    2007-11-09 21h & 0 & 0 & 946.1 & 0.48 & 30.8 & 20.8 & 0 \\
    2007-11-09 18h & 0 & 0 & 945.1 & 9.13 & 36.3 & 16.0 & 0 \\
    2007-11-09 15h & 0 & 0 & 948.4 & 10.13 & 34.5 & 22.4 & 0 \\
    2007-11-09 12h & 0 & 0 & 950.9 & 3.44 & 28.5 & 46.4 & 0 \\
    \bottomrule
\end{tabular}
}

\end{quadro}


\subsubsection{Blocos de dados}

Os dados dos sensores são agrupados em conjuntos denominados `blocos de dados` contendo quatro sensores por bloco, conforme ilustrado na \autoref{fig:payload}. Assim, cada bloco de dados possui 32 bits ($4 \cdot 8$ bits), sendo o valor mínimo de um bloco de dados para montar um datagrama \gls{PTT-A3} (exceto o primeiro bloco que possui apenas 24 bits de comprimento). Para o caso específico em que a PCD irá transmitir apenas um bloco, ela poderá abrigar apenas três sensores. Caso haja mais de um bloco, o comprimento é dado por $24 + (\text{\gls{Nb} - 1}) \cdot 32$ bits \textcite{cnes_services_and_message_formats_ed2_rev2_2006}.

\begin{figure}[H]
	\centering
	\caption{Agrupamento de sensores no payload do datagrama}\label{fig:payload}
	\includegraphics[width=\linewidth]{assets/cap2/payload.pdf}
\end{figure}


\subsection{Tamanho de mensagem}

O campo de tamanho de mensagem \gls{Tm} é utilizado para informar ao receptor quantos blocos de dados estão sendo transmitidos no datagrama, este campo é formado pelo número de blocos em formato binário \gls{Bm} e pelo bit de pariedade \gls{Pm} para fechar uma seção de 4 bits. Como cada bloco possui 32 bits (exceto o primeiro), é possivel determinar o comprimento total dos dados de aplicação de \gls{Bm} = ($N_b - 1)_{2}$. O número de blocos \gls{Nb}, pode variar de um a oito, resultando no compriemnto esperado de bits no receptor para cada caso, conforme o \autoref{quadro:comprimento-mensagem}. 

\begin{quadro}[H]
    \caption{Comprimento em bits para cada tamanho de mensagem ($T_m$)}
    \label{quadro:comprimento-mensagem}
    \small 
    \begin{tabularx}{\textwidth}{>{\centering\arraybackslash}X 
                                  >{\centering\arraybackslash}X 
                                  >{\centering\arraybackslash}X 
                                  >{\centering\arraybackslash}X 
                                  >{\centering\arraybackslash}X}
        \toprule
        \textbf{N° de Blocos} & \textbf{N° de Bits} & \textbf{$B_m$} & \textbf{$P_m$} & \textbf{$T_m$}\\
        \midrule
        1 & 24  & 000 & 0 & 0000\\
        2 & 56  & 001 & 1 & 0011\\
        3 & 88  & 010 & 0 & 0100\\
        4 & 120 & 011 & 1 & 0111\\
        5 & 152 & 100 & 0 & 1000\\
        6 & 184 & 101 & 1 & 1011\\
        7 & 216 & 110 & 0 & 1100\\
        8 & 248 & 111 & 1 & 1111\\
        \bottomrule
    \end{tabularx}
\end{quadro}

Conforme apresentado acima, para cada valor de \gls{Bm}, um valor de \gls{Pm} é calculado. O bit de paridade é calculado de forma a garantir que o número total de bits '1' na mensagem seja par, e é dado por

\vspace{-1em}
\begin{equation}
    P_m = 
    \begin{cases}
    1, & \text{se } \left[ \sum_{i=0}^{B_m} b_i = 0 \right]\mod 2  \\
    0, & \text{se } \left[ \sum_{i=0}^{B_m} b_i = 1 \right]\mod 2 
    \end{cases} \text{.}
\end{equation}

\noindent Ao final, o campo de tamanho de mensagem \gls{Tm} é formado pela concatenação do valor de \gls{Bm} e \gls{Pm}, resultando em um campo de 4 bits \textcite{cnes_services_and_message_formats_ed2_rev2_2006}. 


\subsection{Identificador da PCD}

O identificador da \gls{PCD}, \gls{pcdid}, é um campo de 28 bits presente na estrutura da mensagem do usuário no formato \gls{PTT-A3}. Ele é utilizado para identificar de forma única a \gls{PCD} que está transmitindo a mensagem, sendo essencial para o correto encaminhamento e associação dos dados recebidos no centro de controle do sistema \gls{ARGOS}, \textcite{cnes_services_and_message_formats_ed2_rev2_2006}. 

O \gls{pcdid} é formado por um número de 20 bits, \gls{ipcd}, seguido por oito bits \gls{rpcd} de redundância calculados através da soma (checksum) dos bits do identificador, conforme

\vspace{-1em}
\begin{equation}
R_{PCD} = \left( \sum_{i=0}^{19} I_{PCD} \cdot 2^i \right) \bmod 256 \text{ ,}
\end{equation}
\vspace{-0.2em}
\begin{equation}
    PCD_{ID} = I_{PCD} \oplus R_{PCD} \text{ .}
\end{equation}

\noindent   É importante destacar que a proteção contra erros nesse campo é assegurada de forma indireta pelo uso da codificação convolucional aplicada à mensagem como um todo, além da redundância oferecida pelo número de repetições da mensagem ao longo da passagem do satélite. 


\subsection{Bits de fim de mensagem}

Ao final do datagrama, são inseridos entre sete e nove bits '0' com a finalidade de limpar o registrador do codificador convolucional, dando o encerramento da sequência codificada. A quantidade de bits de fim de mensagem adicionados depende do comprimento total da mensagem do usuário, conforme apresentado na \autoref{quadro:bits-codificacao}. Apesar de não carregar dados úteis a nível de aplicação, esses bits são fundamentais para o correto funcionamento do processo de decodificação \cite{cnes_services_and_message_formats_ed2_rev2_2006}.

\begin{quadro}[H]
    \caption{Comprimento da cauda (bits '0') para cada tamanho de mensagem}
    \label{quadro:bits-codificacao}
    \small 
    \begin{tabularx}{\textwidth}{>{\centering\arraybackslash}X 
                                  >{\centering\arraybackslash}X 
                                  >{\centering\arraybackslash}X 
                                  >{\centering\arraybackslash}X}
        \toprule
        \textbf{N° de Blocos} & \textbf{Bits Aplicação} & \textbf{Bits Datagrama} & \textbf{N° bits "0"} \\
        \midrule
        1 &  24 &  56 & 7 \\
        2 &  56 &  88 & 8 \\
        3 &  88 & 120 & 9 \\
        4 & 120 & 152 & 7 \\
        5 & 152 & 184 & 8 \\
        6 & 184 & 216 & 9 \\
        7 & 216 & 248 & 7 \\
        8 & 248 & 280 & 8 \\
        \bottomrule
    \end{tabularx}
    
\end{quadro}


\section{TRANSMISSOR PTT-A3}\label{sec:transmissor}

Na \autoref{fig:diagrama_transmissor} ilustra-se o diagrama de blocos do transmissor \gls{PTT-A3}. Cada bloco é responsável por uma etapa da transmissão, desde a montagem do datagrama até a modulação em banda passante e a transmissão do sinal \gls{st}.

\begin{figure}[H]
	\centering
	\caption{Diagrama de blocos do transmissor ARGOS-3}\label{fig:diagrama_transmissor}
	\includegraphics[width=\linewidth]{assets/cap2/transmissor.pdf}
    
\end{figure}


\subsection{Codificador convolucional}\label{sec:conv_coding}

Antes da transmissão, os dados do datagrama precisam ser codificados, e esse processo é realizado através de um codificador convolucional. Essa técnica de codificação aplica uma operação lógica sobre uma janela deslizante de bits de entrada \gls{ut}, gerando uma sequência de saída com redundância controlada. Diferente da codificação por bloco, onde os dados são processados em blocos fixos, a codificação convolucional considera a sequência contínua de bits, combinando o bit atual com um número fixo de bits anteriores através de vetores geradores \cite{shu2011error}.

A taxa de codificação utilizada no padrão \gls{PTT-A3} é $R = \text{1/2}$, o que significa que para cada bit de dados de entrada \gls{ut}, são gerados dois bits de saída, um no canal \gls{cI} e outro no canal \gls{cQ}, aumentando a redundância e melhorando a capacidade do sistema de detectar e corrigir erros. Para a codificação convolucional, são utilizados vetores geradores \gls{G0} e \gls{G1}, de acordo com o padrão CCSDS 131.1-G-2 \cite{cnes_services_and_message_formats_ed2_rev2_2006}. A representação binária dos vetores geradores é dada por

\vspace{-1em}
\begin{equation}
    \begin{split}
        G_0 &= 121_{10} \quad \mapsto \quad G_0 = [1, 1, 1, 1, 0, 0, 1] \\
        G_1 &= 091_{10} \quad \mapsto \quad G_1 = [1, 0, 1, 1, 0, 1, 1] \text{.}
    \end{split}
\end{equation}

Os vetores geradores são utilizados para definir a estrutura do registrado do codificação convolucional aplicada à sequência de entrada \gls{ut}, resultando nas saídas \gls{vt0} e \gls{vt1}, que correspondem, respectivamente, aos canais \gls{cI} e \gls{cQ} utilizados posteriormente na modulação \gls{QPSK}. Essa operação pode ser representada por uma multiplicação vetorial entre uma janela deslizante de sete bits da entrada e a matriz formada pelos vetores geradores, conforme

\begin{equation}
\begin{aligned}
    \begin{bmatrix}
        v_t^{(0)} & v_t^{(1)}
    \end{bmatrix}
    &=
    \begin{bmatrix}
        u_{(t)} & u_{(t-1)} & u_{(t-2)} & u_{(t-3)} & u_{(t-4)} & u_{(t-5)} & u_{(t-6)}
    \end{bmatrix} \cdot \\
    &\quad
    \begin{bmatrix}
        1 & 1 & 1 & 1 & 0 & 0 & 1 \\
        1 & 0 & 1 & 1 & 0 & 1 & 1
    \end{bmatrix}^{T}
\end{aligned}
\label{eq:matriz_geradora}
\end{equation}

\noindent que pode ser representada de forma equivalente pelo diagrama de blocos apresentado na \autoref{fig:conv_coding} \cite{cnes_services_and_message_formats_ed2_rev2_2006}.

\begin{figure}[H]
	\centering
	\caption{Diagrama de blocos do codificador convolucional ARGOS-3}
	\label{fig:conv_coding}
	\includegraphics[width=\linewidth]{assets/cap2/conv_coding.pdf}
\end{figure}


\subsection{Embaralhador}

Após a codificação convolucional, os vetores de saída \gls{vt0} e \gls{vt1} são embaralhados para evitar padrões repetitivos, formando os vetores embaralhados \gls{Xn} e \gls{Yn}. O processo de embaralhamento é essencial para aumentar a robustez do sinal contra interferências e ruídos em rajada, pois os dados são espalhados ao longo da transmissão.

O embaralhador é implementado utilizando um registrador de deslocamento de 3 bits para cada canal, conforme apresentado na \autoref{fig:embaralhador}, A cada novo bit de entrada, o registrador é deslocado, e o bit de saida calculado através de uma multiplexação entre as três saidas possiveis. Desta forma o index de embaralhamento de uma sequência de entrada é dado por

\vspace{-1em}
\begin{equation}
    \begin{bmatrix}
        X_1 & X_2 & X_3 & X_4 & X_5 & X_6 & \dots \\
        Y_1 & Y_2 & Y_3 & Y_4 & Y_5 & Y_6 & \dots \\
    \end{bmatrix}
    \mapsto 
    \begin{bmatrix}
        Y_1 & X_2 & Y_2 & Y_4 & X_5 & Y_5 & \dots \\
        X_1 & X_3 & Y_3 & X_4 & X_6 & Y_6 & \dots \\ 
    \end{bmatrix}
\end{equation}

\noindent Onde \gls{Xn} e \gls{Yn} são os vetores de bits embaralhados. O embaralhador é utilizado para garantir que os dados sejam melhor distribuídos diminuindo a correlação entre os bits de forma a aumentar a aleatoriedade antes da modulação \gls{QPSK}, o que permite atingir uma melhor característica espectral. Esse processo pode ser representado pelo diagrama de blocos apresentado na \autoref{fig:embaralhador} \cite{cnes_services_and_message_formats_ed2_rev2_2006, rodrigues_demodulador_2018}.

\begin{figure}[H]
	\centering
	\caption{Embaralhador de dados para o ARGOS-3}\label{fig:embaralhador}
	\includegraphics[width=\linewidth]{assets/cap2/scrambler.pdf}
\end{figure}

\subsection{Codificação de Linha}\label{sec:line_coding}

Em seguida, os vetores de bits embaralhados \gls{Xn} e \gls{Yn}, já multiplexados com os vetores de preâmbulo \gls{SIn} e \gls{SQn},precisam ser convertidos em vetores de simbolos, para isso é aplicada uma codificação de linha. O vetor \gls{Xn} é codificado utilizando a técnica \gls{NRZ}, enquanto o vetor \gls{Yn} é codificado utilizando a técnica \gls{Manchester} \cite{cnes_services_and_message_formats_ed2_rev2_2006}.

A codificação \gls{NRZ} é realizada removendo a componente DC da sequência de entrada, ou seja, no vetor \gls{Xn} cada bit '1' é representado por '+1' e cada bit '0' é representado por '-1'. Esse processo pode ser descrito pela expressão

\begin{equation}
I[n] = 
\begin{cases}
+1, +1, & \text{se } X[n] = 1 \\
-1, -1, & \text{se } X[n] = 0 \text{ ,}
\end{cases}
\end{equation}

\noindent resultando em um vetor de saída \gls{In}, neste caso com o dobro do comprimento do vetor de entrada (para manter o mesmo comprimento do vetor \gls{Qn}). 

Para o vetor \gls{Yn}, é aplicada a codificação \gls{Manchester}, onde cada bit de entrada é representado por dois simbolos de saída, alternando entre '-1' e '+1'.  A codificação \gls{Manchester} é utilizada para garantir que haja transições de nível no sinal mesmo em sequências longas de bits iguais, essas transições refletem em mais trocas de simbolos na modulação \gls{QPSK}. Esse processo pode ser descrito pela expressão

\begin{equation}
Y[n] = 
\begin{cases}
+1,-1, & \text{se } Y[n] = 1 \\
-1,+1, & \text{se } Y[n] = 0 \text{ ,}
\end{cases}
\end{equation}

\noindent resultando assim em um vetor de saída \gls{Qn} codificado em Manchester.
\begin{figure}[H]
	\centering
	\caption{Codificação de linha dos vetores I e Q}\label{fig:line_coding}
	\includegraphics[width=\linewidth]{assets/cap2/example_encoder_time.pdf}
\end{figure}


\subsection{Modulação de Pulso}\label{sec:pulse_modulation}

Uma vez com os vetores de simbolo \gls{In} e \gls{Qn}, é aplicada a modulação de pulso, isto é, os vetores de simbolo são superamostrados, em função de \gls{fs} e filtrados para formar uma sequência contínua de símbolos ao longo de \gls{t}, onde cada bit de informação é transmitido durante um período de tempo $\text{ \gls{Tb}} = 1/ \text{\gls{Rb}}$ (tempo de bit), definido com base na taxa de bit \gls{Rb} \cite{cnes_services_and_message_formats_ed2_rev2_2006}.


Para aplicar a modulação de pulso e gerar os sinais analógicos \gls{dI} e \gls{dQ}, os vetores de simbolo são multiplicados com um pulso com resposta ao impulso \gls{gt}. A formatação é dada por 

\vspace{-1em}
\begin{equation}
    \begin{array}{c@{\quad\text{e}\quad}c}
        d_I'(t) = \sum_{n} I[n] \cdot g(t - nT_b) &
        d_Q'(t) = \sum_{n} Q[n] \cdot g(t - nT_b)
    \end{array} \text{ ,}
\end{equation}

\noindent onde é utilizado um pulso \gls{gt} do tipo (\gls{RRC}), definido por

\begin{equation}
    g(t) = \frac{(1-\alpha) \text{sinc}((1- \alpha) t / T_b) + \alpha(4/\pi) \cos(\pi(1 + \alpha)t/T_b) }{1 - (4\alpha )^2}
\end{equation}

\noindent onde \gls{alpha} é o fator de roll-off do pulso, que controla a largura de banda \gls{W} do sinal modulado. 

A formatação dos vetores \gls{In} e \gls{Qn} é ilustrada na \autoref{fig:pulse_modulate}, onde a resposta ao impulso \gls{gt} é apresentada em conjunto com os sinais \gls{dI} e \gls{dQ}. 

\begin{figure}[H]
	\caption{Modulação de pulso dos canais $I$ e $Q$}\label{fig:pulse_modulate}
	\includegraphics[width=\linewidth]{assets/cap2/example_formatter_comp_rrc.pdf}
\end{figure}

Quanto maior o \gls{alpha}, mais suave é a transição entre os símbolos, mas também maior é a largura de banda ocupada \cite{10555531840}. A \autoref{fig:formatter_comp} apresenta a comparação entre os sinais \gls{dI} e \gls{dQ} quando é aplicada a modulação de pulso com pulso \gls{RRC} e \gls{Rect}, ou seja, quando o vetor de simbolos é apenas superamostrado. 

\begin{figure}[H]
	\caption{Comparação entre $I$ e $Q$ com e sem modulação de pulso}\label{fig:formatter_comp}
	\includegraphics[width=\linewidth]{assets/cap2/example_formatter_comp.pdf}
\end{figure}

Nota-se que utilizando formatação \gls{Rect}, o sinal apresenta transições abruptas entre os símbolos, o que pode gerar interferência entre símbolos adjacentes \gls{ISI} e aumentar a largura de banda do sinal transmitido, a resposta ao impulso do pulso \gls{Rect} é dada por

\vspace{-1em}
\begin{equation}
    g(t) = \text{rect}\left(\frac{t}{T_b}\right)
\end{equation}

\noindent onde, \gls{gt} é a resposta ao impulso da função, \gls{t} é o vetor de tempo e \gls{Tb} é o período de bit. Já com a modulação de pulso, o sinal é suavizado no dominio do tempo, reduzindo a interferência entre símbolos e ajustando a largura de banda do sinal conforme os parâmetros do modulador de pulso \cite{cnes_services_and_message_formats_ed2_rev2_2006,10555531840,rodrigues_demodulador_2018}.

\subsection{Modulação em banda passante} 

Uma vez com os pares de sinal \gls{dI} e \gls{dQ} correspondentes a cada canal, pode-se realizar a modulação em banda passante para transmissão do sinal \gls{st}. Na \autoref{fig:qpsk_modulator} é ilustrado o diagrama de blocos do modulador \gls{QPSK} \cite{cnes_services_and_message_formats_ed2_rev2_2006}.

\begin{figure}[H]
	\centering
	\caption{Diagrama de blocos do modulador QPSK}\label{fig:qpsk_modulator}
	\includegraphics[width=\linewidth]{assets/cap2/qpsk_modulator.pdf}
\end{figure}

No processo de modulação QPSK, o sinal em fase \gls{dI} é multiplicado por uma componente cossenoidal em frequência \gls{fc} e o sinal em quadratura \gls{dQ} é multiplicado por uma componente senoidal em \gls{fc}, seguindo a expressão

\vspace{-0.8em}
\begin{equation}
    s(t) = A d_I'(t) \cos(2\pi f_c t + \phi_0) - Ad_Q'(t) \sin(2\pi f_c t + \phi_0) \text{ ,}
\end{equation}

\noindent onde \gls{A} é a amplitude do sinal modulado, \gls{fc} é a frequência da portadora, em torno de $401,625$ \gls{mhz} e \gls{phi0} é o desvio de fase inicial do sinal, que é considerado como zero ($\phi_0 = 0$). A \autoref{fig:iq_modulation} ilustra o processo de modulação IQ, mostrando os sinais \gls{dI} e \gls{dQ} e a constelação resultante, onde cada ponto representa um símbolo modulado.

\begin{figure}[H]
	\centering
	\caption{Modulação em banda passante dos canais $I$ e $Q$}\label{fig:iq_modulation}
	\includegraphics[width=\linewidth]{assets/cap2/transmitter_modulator_time_comp.pdf}
\end{figure}

A constelação \gls{QPSK} na \autoref{fig:iq_modulation} mapea cada par de amostra dos vetores \gls{dI} e \gls{dQ} no plano complexo ($I + jQ$). Após a transmissão, o sinal \gls{st} é somado a um vetor de ruído \gls{rt} para simular as condições reais no momento da recepção.

\section{RECEPTOR PTT-A3}\label{sec:receptor}

Como a transmissão dos dados das \gls{PCD} não segue uma estrutura de canais discretos, ou seja, cada \gls{PCD} seleciona uma frequência \gls{fn} dentro da faixa de $401,62$ a $401,65MHz$ e realiza a transmissão. Dessa forma, o receptor no satélite necessita de um mecanismo de detecção de portadora para avaliar as candidatas de \gls{fc}, podendo assim demodular o sinal.

\begin{figure}[H]
	\centering
	\caption{Diagrama de blocos do receptor ARGOS-3}\label{fig:diagrama_receptor}
	\includegraphics[width=\linewidth]{assets/cap2/receptor.pdf}
\end{figure}


\subsection{Detecção de portadora}

Para realizar a detecção da portadora, o sinal recebido acrescido de ruído deve ser inicialmente amostrado e dividido em segmentos discretos \gls{xn} no tempo. O satélite \gls{ARGOS} realiza a decisão do sinal a cada $10 \text{ms}$, esse processo é definido por 

\vspace{-0.4em}
\begin{equation}
    x_n[m] = s(mT_n) \text{ ,}
\end{equation}

\noindent onde \gls{xn} representa o segmento de sinal, \gls{st} é o sinal recebido, \gls{m} é o índice de amostra do segmento de tempo e \gls{Tn} é o tempo de amostragem, definido como $T_n = f_s * 10 \text{ms}$, onde \gls{fs} é a frequência de amostragem do sistema. 


Em seguida, aplica-se a \gls{FFT} no vetor \gls{xn}, obtendo-se o espectro de frequência do sinal amostrado \gls{xnk}, conforme a expressão 

\vspace{-0.4em}
\begin{equation}
    X_n[k] = \sum_{m=0}^{N-1} x_n[m]\, e^{-j2\pi km/N} \text{ .}
\end{equation}

A partir da amostras de \gls{xnk}, se calcula a potência em cada componente de frequência, para obter os valores \gls{Pnk}, conforme

\vspace{-0.4em}
\begin{equation}
    P_n[k] = |X_n[k]|^2 \text{ .}
\end{equation}

Em seguida, para cada índice \gls{k} do espectro calculado, é feita uma comparação com um limiar pré-definido \gls{Pt}, conforme foi apresentado anteriormente na \autoref{fig:portadora_pura_freq}. Caso a potência \gls{Pnk} seja maior que \gls{Pt}, a frequência é registrada pois existe a possibilidade de que uma portadora \gls{fn} esteja presente naquela frequência. 

Por fim, no proximo segmento, $X_{n+1}[k]$, o sistema verifica se a frequência registrada no segmento anterior persiste, ou seja, se a potência de $k$ também é maior que o limiar \gls{Pt}, caso positivo, o sistema considera que a frequência \gls{fn} é a portadora \gls{fc} do sinal recebido e instância a cadeia de recepção passando \gls{fc} como parâmetro.



\subsection{Demodulador banda base}

Para realizar a demodulação do sinal \gls{st} recebido retornando as componentes \gls{dXprime} e \gls{dYprime} em banda base, o sinal \gls{st} é multiplicado por duas portadoras ortogonais, \gls{xi} para demodular o canal \gls{cI} e \gls{yq} para demodular o canal \gls{cQ}. Assumindo sincronismo perfeito, o processo de demodulação para o canal \gls{cI} pode ser expresso como

\vspace{-1.2em}
\begin{equation}
d_X'(t) = s(t) \cdot x_I(t) = \left[A \cdot d_I'(t) \cos(2\pi f_c t ) - A \cdot d_Q'(t) \sin(2\pi f_c t )\right] \cdot 2\cos(2\pi f_c t )
\end{equation}
\vspace{-0.8em}
\begin{equation}
    d_X'(t) =
    \underbrace{A \cdot d_I'(t)}_{\text{Banda base}} + 
    \underbrace{\left[
        A \cdot d_I'(t) \cos(4\pi f_c t ) 
        - A \cdot d_Q'(t) \sin(4\pi f_c t )
    \right]}_{\text{Dobro da frequência $f_c$}} \text{ ,}
\end{equation}

\noindent onde \gls{dI} é o sinal banda base. O mesmo processo é realizado para o canal \gls{cQ}, que isola o sinal em quadratura \gls{dQ} da seguinte forma

\vspace{-1.2em}
\begin{equation}
d_Y'(t) = s(t) \cdot y_Q(t) = \left[A \cdot d_I'(t) \cos(2\pi f_c t ) - A \cdot d_Q'(t) \sin(2\pi f_c t )\right] \cdot 2\sin(2\pi f_c t )
\end{equation}
\vspace{-0.8em}
\begin{equation}
    d_Y'(t) =  
    \underbrace{A \cdot d_Q'(t)}_{\text{Banda base}} + 
    \underbrace{\left[
        A \cdot d_Q'(t) \cos(4\pi f_c t ) 
        + A \cdot d_I'(t) \sin(4\pi f_c t )
    \right]}_{\text{Dobro da frequência $f_c$}} \text{ .}
\end{equation}

A \autoref{fig:iq_demodulation} ilustra o processo de demodulação dos canais \gls{cI} e \gls{cQ}, onde os sinais multiplicados por \gls{xi} e \gls{yq} são apresentados no espectro, mostrando a presença dos sinais em banda base e as componentes de alta frequência resultantes da multiplicação com as portadoras (considerando um $f_c = 2 \text{\gls{khz}}$). \cite{cnes_services_and_message_formats_ed2_rev2_2006}

\begin{figure}[H]
	\centering
	\caption{Componentes $I$ e $Q$ após demodulação}\label{fig:iq_demodulation}
	\includegraphics[width=\linewidth]{assets/cap2/receiver_demodulator_freq.pdf}
\end{figure}


\subsection{Filtragem passa baixa}\label{sec:filtering}

Para isolar os sinais em banda base \gls{dIprime} e \gls{dQprime}, é necessário aplicar um \gls{LPF}, para isso utiliza-se um filtro Butterworth (6ª Ordem) com frequência de corte de $0{,}6$ \gls{khz}, com resposta ao impulso \gls{ht}, que remove as componentes de alta frequência em $2 \cdot \text{\gls{fc}}$ resultantes da multiplicação de demodulação realizada anteriormente \cite{rodrigues_demodulador_2018}.

O filtro Butterworth é escolhido por sua resposta em frequência plana e sem ondulações (ganhos ou atenuações) na banda passante, o que é ideal para preservar a integridade do sinal que está passando pelo filtro, neste caso como é um filtro passa baixa, o sinal que está em banda base. A aplicação do filtro \gls{LPF} aos sinais \gls{dXprime} e \gls{dYprime} pode ser expressa como

\vspace{-1em}
\begin{equation}
    \begin{array}{c@{\quad\text{e}\quad}c}
        d_I'(t) = d_X'(t) * h(t) &
        d_Q'(t) = d_Y'(t) * h(t)
    \end{array} \text{ ,}
\end{equation}

\noindent onde $*$ representa a operação de convolução entre os sinais \gls{dXprime} e \gls{dYprime} com a resposta ao impulso do filtro \gls{ht}. A resposta ao impulso do filtro \gls{ht} é dada por

\vspace{-1em}
\begin{equation}
    h(t) = \mathcal{F}^{-1}\{H(f)\} \text{ ,}
\end{equation}

\noindent onde $\mathcal{F}^{-1}$ é a transformada inversa de Fourier e \gls{Hf} é a resposta em frequência do filtro Butterworth, definida por

\vspace{-1em}
\begin{equation}
    H(f) = \frac{1}{\sqrt{1 + \left(\frac{f}{f_c}\right)^{2n}}} \text{ ,}
\end{equation}

\noindent onde \gls{n} é a ordem do filtro (neste caso, $n=6$) e \gls{fc} é a frequência de corte do filtro, definida como $0,6$ \gls{khz}. Os polos e zeros do filtro Butterworth também podem ser calculados com base na função de transferência \gls{Hs} do filtro, conforme apresentado na \autoref{fig:butterworth_poles} \cite{10555531840}.

\begin{figure}[H]
    \centering
    \caption{Polos e zeros do filtro Passa Baixa}\label{fig:butterworth_poles}
    \includegraphics[width=0.8\linewidth]{assets/cap2/example_lpf_pz.pdf}
\end{figure}

A \autoref{fig:lowpassfilter} apresenta o espectro dos sinais \gls{dXprime} e \gls{dYprime} antes e após a filtragem passa-baixa, bem como a resposta em frequência \gls{Hf} do filtro, onde é possível observar a remoção das componentes de alta frequência, deixando apenas os sinais em banda base, além da frequência de corte do filtro (onde a atenuação alcança $-3$ \gls{db}).

\begin{figure}[H]
	\centering
	\caption{Componentes $I$ $Q$ após filtragem passa baixa}\label{fig:lowpassfilter}
	\includegraphics[width=\linewidth]{assets/cap2/receiver_lpf_freq.pdf}
\end{figure}

\subsection{Filtragem casada}

Os sinais \gls{dIprime} e \gls{dQprime} em banda base, e já filtrados por um \gls{LPF}, passam por um filtro casado \gls{MF}, que é utilizado para maximizar a relação sinal-ruído (\gls{SNR}). O filtro casado é ajustado para coincidir com o inverso do pulso  \gls{gti} utilizado na formatação do sinal na transmissão \cite{10555531840}. 

O filtro casado é aplicado aos sinais \gls{dIprime} e \gls{dQprime} através da expressão

\vspace{-1em}
\begin{equation}
    \begin{array}{c@{\quad\text{e}\quad}c}
        I'(t) = d_I(t) * g(-t) &
        Q'(t) = d_Q(t) * g(-t)
    \end{array} \text{ .}
\end{equation}


A \autoref{fig:matchedfilter} apresenta o espectro dos sinais \gls{Iprime} e \gls{Qprime} após a filtragem casada, onde é possível observar a melhoria na relação sinal-ruído e a remoção de componentes indesejadas. 

\begin{figure}[H]
	\centering
	\caption{Componentes $I$ e $Q$ após filtragem casada}\label{fig:matchedfilter}
	\includegraphics[width=\linewidth]{assets/cap2/example_mf_time_comp.pdf}
\end{figure}


\subsection{Decisão de simbolos}

Uma vez com os sinais \gls{dXprime} e \gls{dYprime} filtrados, considerando sincronismo perfeito, o próximo passo é a decisão dos sinais em instantes de \gls{Tb} (instante ótimo de decisão) a partir de um delay inicial \gls{tau}, para recuperar os símbolos transmitidos. O processo de decisão pode ser expresso como

\vspace{-1em}
\begin{equation}
    \begin{array}{c@{\quad\text{e}\quad}c}
        I'[n] = d'_I(n \cdot T_b + \tau) & Q'[n] = d'_Q(n \cdot T_b + \tau)
    \end{array} \text{ ,}
\end{equation}

\noindent onde \gls{n} é o índice de amostra, \gls{Tb} é o tempo de bit $1/R_b$ e \gls{tau} é o atraso inicial da decisão. A \autoref{fig:sampling} apresenta os sinais amostrados \gls{Inprime} e \gls{Qnprime}. 

\begin{figure}[H]
	\centering
	\caption{Amostragem das componentes $I$ e $Q$}\label{fig:sampling}
	\includegraphics[width=\linewidth]{assets/cap2/example_sampler_time.pdf}
    
\end{figure}

Os valores amostrados \gls{Inprime} e \gls{Qnprime} são então decididos, isto é, quantizados para valores discretos, correspondendo aos símbolos transmitidos. O mapeamento dos pares pode ser expressado como

\begin{equation} 
    I'[n] = \begin{cases}
    +1, & \text{se } I'[n] \geq 0 \\
    -1, & \text{se } I'[n] < 0
    \end{cases}, \quad
    Q'[n] = \begin{cases}
    +1, & \text{se } Q'[n] \geq 0 \\
    -1, & \text{se } Q'[n] < 0
    \end{cases} \text{ ,}
\end{equation}

\noindent onde \gls{Inprime} e \gls{Qnprime} são os vetores simbolo. A \autoref{fig:comparacao_bits} apresenta os simbolos decididos nos vetores \gls{Inprime} e \gls{Qnprime} após o processo de quantização, em comparação com os vetores originais \gls{In} e \gls{Qn} transmitidos.

\begin{figure}[H]
	\centering
	\caption{Comparação dos vetores $I$ e $Q$ decididos com vetores transmitidos}\label{fig:comparacao_bits}
	\includegraphics[width=\linewidth]{assets/cap2/comparacao_bits.pdf}
\end{figure}

\subsection{Decodificador de linha}

Após a decisão dos símbolos, os vetores \gls{Inprime} e \gls{Qnprime} precisam ser convertidos de volta para a representação original dos bits. Para isso, são aplicadas as técnicas de decodificação de linha inversas às utilizadas na transmissão: \gls{NRZ} para o canal \gls{cI} e \gls{Manchester} para o canal \gls{cQ}. A decodificação \gls{NRZ} é realizada mapeando os valores '+1' para o bit '1' e '-1' para o bit '0', conforme

\begin{equation}
    X'[n] = 
    \begin{cases}
    1, & \text{se } I'[n] = +1 \\
    0, & \text{se } I'[n] = -1
    \end{cases} \text{ ,}
\end{equation}


\noindent resultando no vetor de bits \gls{Xnprime} correspondente ao canal \gls{cI}. O mesmo processo é aplicado ao canal \gls{cQ}, onde a decodificação \gls{Manchester} mapeia os pares de valores '+1,-1' para o bit '1' e '-1,+1' para o bit '0', conforme

\begin{equation}
    Y'[n] = 
    \begin{cases}
    1, & \text{se } Q'[n] = +1,-1 \\
    0, & \text{se } Q'[n] = -1,+1
    \end{cases} \text{ ,}
\end{equation}

\noindent resultando no vetor de bits \gls{Ynprime} correspondente ao canal \gls{cQ}. 

\subsection{Desembaralhador}

Após o processo de decodificação de linha, os dados vetores de bits \gls{Xnprime} e \gls{Ynprime} ainda encontram-se embaralhados, então é necessário realizar o desembaralhamento para restaurar a sequência original dos bits antes da decodificação convolucional \cite{cnes_services_and_message_formats_ed2_rev2_2006}.

O desembaralhador utiliza a mesma estrutura lógica do embaralhador, porém executando a operação inversa de reorganização dos bits, com base em regras posicionais dependentes do índice dos bits \gls{n}. A \autoref{fig:unscrambler} apresenta o diagrama de blocos do desembaralhador utilizado no \gls{PTT-A3}, que refaz a ordenação dos bits após a demodulação \cite{rodrigues_demodulador_2018}.

\begin{figure}[H]
\centering
\caption{Diagrama de blocos do Desembaralhador ARGOS-3}\label{fig:unscrambler}
\includegraphics[width=\linewidth]{assets/cap2/unscrambler.pdf}
\end{figure}

O processo de desembaralhamento retorna os vetores de bit, \gls{vt0prime} e \gls{vt1prime}, que correspondem aos dados codificados nos canais \gls{cI} e \gls{cQ}, respectivamente.

\subsection{Decodificador convolucional}

Após o desembaralhamento, os vetores \gls{vt0prime} e \gls{vt1prime} estão prontos para a decodificação convolucional para obter novamente a sequência \gls{utprime} correspondente ao vetor de bits do datagrama. O algoritmo mais popularmente utilizado para essa etapa é o algoritmo de Viterbi, que implementa a decodificação \gls{MLD} para códigos convolucionais, calculando o caminho mais provável através de \gls{Hamming} para a mensagem recebida em relação à mensagem possível \cite{cnes_services_and_message_formats_ed2_rev2_2006, rodrigues_demodulador_2018}.

No padrão \gls{ARGOS-III}, o código convolucional utilizado segue o padrão \gls{CCSDS} 131.1-G-2, que, por sua vez, possui uma distância livre conhecida de $\text{\gls{dfree}} = 10$. Quanto maior for o valor de \gls{dfree}, maior será a robustez do código e sua capacidade de detectar e corrigir erros.

\noindent Assim, conforme o comparativo ilustrado na \autoref{fig:curva_bersnr} entre a transmissão \gls{QPSK} sem codificação e com codificação convolucional, o uso desta técnica na transmissão do sinal permite operar em valores de \gls{EbN0} menores em relação a não utilização do codificador, para a \gls{BER} visada pelo sistema, permitindo uso em ambientes com maior ruído para a mesma taxa de erro desejada.

\begin{figure}[H]
	\centering
	\caption{Comparação de BER vs \gls{EbN0} utilizando codificação convolucional}\label{fig:curva_bersnr}
	\includegraphics[width=\linewidth]{assets/cap2/bersnr.pdf}
\end{figure}