\chapter{Introdução}\label{cap:introducao}


A introdução abre o trabalho propriamente dito. Tem a finalidade de apresentar os motivos que levaram o autor a realizar a pesquisa, o problema abordado, os objetivos e a justificativa. O objetivo principal da introdução é situar o leitor no contexto da pesquisa. O leitor deverá perceber claramente o que foi analisado, como e por que, as limitações encontradas, o alcance da investigação e suas bases teóricas gerais. Ela tem, acima de tudo, um caráter didático de apresentar o que foi investigado, levando-se em conta o leitor a que se destina e a finalidade do trabalho~\cite{ifsc:manual:normalizacao}. 
asdasdsad
Assim, na introdução contextualize o tema, delimite o assunto, apresente um rápido histórico do problema e das soluções porventura já apresentadas, com breve revisão crítica das investigações anteriores; faça referência às fontes de material, aos métodos seguidos, às teorias ou aos conceitos que embasam o desenvolvimento e a argumentação, às eventuais faltas de informação, ao instrumental utilizado. A introdução deverá conter, ainda:

\begin{itemize}
   \item Justificativa -- trata-se da relevância, o motivo pelo qual tal pesquisa deve ser realizada. Justifica-se aqui a escolha do tema, a delimitação feita e a relação que o pesquisador possui com ele. Procura-se demonstrar a legitimidade, a pertinência, o interesse e a capacidade do pesquisador em lidar com o referido tema. Deve-se fazer o mesmo em relação ao problema e à hipótese, mostrando a relevância científica do tema para o pesquisador. Deve-se fazer, então, nesta parte, a justificativa para o tema, para o problema e para a hipótese, nos termos em que foram formulados na fase de elaboração do projeto de pesquisa;
   
   \item Definição do problema -- um problema decorre de um aprofundamento do tema. Ele deve delimitar a pesquisa. Diversos autores sugerem que o problema deve ter algumas características, tais como: a) deve ser formulado como pergunta - isso facilita sua identificação por quem consulta o projeto de pesquisa; b) deve ser claro e preciso; c) deve ser delimitado a uma dimensão variável, pois muitas vezes, o problema é formulado de uma maneira muito ampla, impossível de ser investigado 
   
   \item Objetivo geral e objetivos específicos -- detalhado dentro das seções abaixo.
\end{itemize}

\section{Objetivos}

Neste item deverá ser indicado claramente o que se deseja fazer, o que se pretende alcançar. É fundamental que estes objetivos sejam possíveis de serem atingidos. Geralmente se formula um objetivo geral articulando-o a outros objetivos mais específicos.

\subsection{Objetivo geral}

Procura-se determinar\footnote{Atenção! Inicie a frase com um verbo abrangente e no infinitivo, como: compreender, saber, avaliar, verificar, constatar, analisar, desenvolver, conhecer, entender, levantar, mapear, identificar.}, com clareza e objetividade, o seu propósito com a realização da pesquisa. Deve-se estar atento ao fato de que nesta pesquisa, em nível de graduação ou pós-graduação, os propósitos são essencialmente acadêmicos, como mapear, identificar, levantar, diagnosticar, traçar o perfil ou historiar determinado assunto específico dentro de um tema. Um objetivo bem redigido explica o quê, com o quê (quem), por meio de quê, onde, quando sobre a pesquisa.

\subsection{Objetivos específicos}

Significa aprofundar as intenções expressas no objetivo geral. Propõe-se mapear, identificar, levantar, diagnosticar, traçar o perfil ou historiar determinado assunto específico dentro de um tema. Assim, para elaborar os objetivos específicos deve-se:

\begin{itemize}
   \item detalhar o objetivo geral mostrando o que se pretende alcançar com a pesquisa;
   \item tornar operacional o objetivo geral, indicando exatamente o que será realizado na pesquisa;
   \item usar verbos que admitam poucas interpretações e no infinitivo, como: identificar, caracterizar, comparar, testar, aplicar, observar, medir, localizar, selecionar, distinguir.
\end{itemize}


% \section{Organização do texto}

% O texto está organizado da seguinte forma: No \autoref{cap:revisao} é apresentado um pouco mais de como fazer um outro capítulo, apresentando ainda formas para inserir figuras. No \autoref{cap:proposta} é apresentado uma forma para adicionar uma tabela. Por fim, no \autoref{cap:conclusoes} são apresentadas as conclusões sobre este trabalho.