% R_b taxa de bits
\glsxtrnewsymbol[description={\textit{Rate bit}, 400 bps (ARGOS-3)}]
{Rb}
{\ensuremath{R_b}}

% T_b tempo de bit
\glsxtrnewsymbol[description={Tempo de bit, $T_b = 1/R_b$}]
{Tb}
{\ensuremath{T_b}}

% Bps bits per second
\glsxtrnewsymbol[description={\textit{Bits per second}}]
{bps}
{\ensuremath{bps}}

% f_c carrier frequency
\glsxtrnewsymbol[description={\textit{Carrier frequency}, $\approx 401.65\,\text{MHz}$ (ARGOS-3)}]
{fc}
{\ensuremath{f_c}}

% f_n nominal frequency
\glsxtrnewsymbol[description={Frequência randômica selecionada pela PCD, $401.620 \leq f_n \leq 401.650\,\text{MHz}$ (ARGOS-3)}]
{fn}
{\ensuremath{f_n}}

% x_n - amostra do sinal no tempo
\glsxtrnewsymbol[description={Segmento de sinal de s(t) amostrado no tempo}]
{xn}
{\ensuremath{x_n[m]}}

% XnK - espectro de frequência do segmento de sinal
\glsxtrnewsymbol[description={Espectro de frequência do segmento de sinal}]
{xnk}
{\ensuremath{X_n[k]}}

% k - índice de frequência
\glsxtrnewsymbol[description={Índice de frequência do espectro de frequência do segmento de sinal}]
{k}
{\ensuremath{k}}

% Pnk - potência do espectro de frequência do segmento de sinal
\glsxtrnewsymbol[description={Potência do espectro de frequência do segmento de sinal}]
{Pnk}
{\ensuremath{P_n[k]}}

% Pt - potência de treashold
\glsxtrnewsymbol[description={Valor de potência de \textit{threshold} para detecção de portadora}]
{Pt}
{\ensuremath{P_t}}

% m - índice de amostra
\glsxtrnewsymbol[description={Índice de amostra do segmento de sinal}]
{m}
{\ensuremath{m}}

% T_s - tempo de decisão
\glsxtrnewsymbol[description={Tempo de decisão, $T_s = 10\,\text{ms}$}]
{Ts}
{\ensuremath{T_s}}

% T_n - Index de tempo do segmento
\glsxtrnewsymbol[description={Index do segmento de sinal}]
{Tn}
{\ensuremath{n}}

% xi - componente cossenoidal utilizada na modulação do , canal $I$
\glsxtrnewsymbol[description={Portadora cossenoidal utilizada na modulação do, canal $I$}]
{xi}
{\ensuremath{x_i(t)}}

% yq - componente senoidal utilizada na modulação do, canal $Q$
\glsxtrnewsymbol[description={Portadora senoidal utilizada na modulação do, canal $Q$}]
{yq}
{\ensuremath{y_q(t)}}

% BER - taxa de erro de bit
\glsxtrnewsymbol[description={\textit{Bit Error Rate}}]
{BER}
{\ensuremath{BER}}

% EbN0 - razão energia por bit e densidade espectral de potência do ruído
\glsxtrnewsymbol[description={\textit{Energy per bit to noise power spectral density ratio}}]
{EbN0}
{\ensuremath{E_b/N_0}}

% MHz megahertz
\glsxtrnewsymbol[description={\textit{Megahertz}}]
{mhz}
{\ensuremath{MHz}}

% kHz kilohertz
\glsxtrnewsymbol[description={\textit{Kilohertz}}]
{khz}
{\ensuremath{kHz}}

% Sn - Preamble sequence
\glsxtrnewsymbol[description={Sequêcia de preâmbulo, $S[n] = 2BEEEEBF_{16}$, (ARGOS-3)}]
{Sn}
{\ensuremath{S[n]}}

% SIn - Preamble sequence channel I
\glsxtrnewsymbol[description={Sequência de preâmbulo do, canal $I$}]
{SIn}
{\ensuremath{S_I[n]}}

% SQn - Preamble sequence channel Q
\glsxtrnewsymbol[description={Sequência de preâmbulo do, canal $Q$}]
{SQn}
{\ensuremath{S_Q[n]}}

% número de blocos
\glsxtrnewsymbol[description={Número de blocos, $0 < N_b < 9$ (ARGOS-3)}]
{Nb}
{\ensuremath{N_b}}

% bits de mensagem
\glsxtrnewsymbol[description={Sequência de bits do comprimento de mensagem, $(0 \leq T_m < 8)_{10}$ (ARGOS-3)}]
{Bm}
{\ensuremath{B_m}}

% bit de pariedade
\glsxtrnewsymbol[description={Bit de paridade do campo $B_m$}]
{Pm}
{\ensuremath{P_m}}

% tamanho da mensagem
\glsxtrnewsymbol[description={Tamanho da mensagem}]
{Tm}
{\ensuremath{T_m}}

% Ipcd - Identificador da PCD
\glsxtrnewsymbol[description={Número da PCD, $(0 \leq I_{PCD} < 1048575)_{10}$ (ARGOS-3)}]
{ipcd}
{\ensuremath{I_{PCD}}}

% Rpcd - Bits de checksum do Identificador da PCD
\glsxtrnewsymbol[description={Bits de checksum do Identificador da PCD}]
{rpcd}
{\ensuremath{R_{PCD}}}

% pcdid - Identificador da PCD + checksum
\glsxtrnewsymbol[description={Identificador da PCD, número da PCD + bits de checksum}]
{pcdid}
{\ensuremath{PCD_{ID}}}

% Sinal modulado
\glsxtrnewsymbol[description={Sinal modulado em banda passante, em função de tempo $t$}]
{st}
{\ensuremath{s(t)}}

% t - tempo
\glsxtrnewsymbol[description={Vetor de Tempo}]
{t}
{\ensuremath{t}}

% rt - ruído adicionado ao sinal modulado
\glsxtrnewsymbol[description={Ruído $AWGN$, em função de tempo $t$}]
{rt}
{\ensuremath{r(t)}}

% f_s - frequência de amostragem
\glsxtrnewsymbol[description={Frequência de amostragem, $f_s = 128\,\text{KHz}$}]
{fs}
{\ensuremath{f_s}}

% dI - sinal modulado, canal $I$
\glsxtrnewsymbol[description={Sinal modulado em banda base, canal $I$}]
{dI}
{\ensuremath{d_I(t)}}

% dQ - sinal modulado, canal $Q$
\glsxtrnewsymbol[description={Sinal modulado em banda base, canal $Q$}]
{dQ}
{\ensuremath{d_Q(t)}}

% dI' - sinal demodulado, canal $I$
\glsxtrnewsymbol[description={Sinal demodulado em banda base, canal $I$}]
{dIprime}
{\ensuremath{d_I'(t)}}

% dQ' - sinal demodulado, canal $Q$
\glsxtrnewsymbol[description={Sinal demodulado em banda base, canal $Q$}]
{dQprime}
{\ensuremath{d_Q'(t)}}

% dX't - componente demodulada correspondente ao, canal $I$
\glsxtrnewsymbol[description={Sinal filtrado (filtragem passa baixa), canal $I$}]
{dXt}
{\ensuremath{d_X'(t)}}

% dY't - componente demodulada correspondente ao, canal $Q$
\glsxtrnewsymbol[description={Sinal filtrado (filtragem passa baixa), canal $Q$}]
{dYt}
{\ensuremath{d_Y'(t)}}

% I' - sinal demodulado, canal $I$
\glsxtrnewsymbol[description={Sinal filtrado (filtragem casada), canal $I$}]
{Iprime}
{\ensuremath{I'(t)}}

% Q' - sinal demodulado, canal $Q$
\glsxtrnewsymbol[description={Sinal filtrado (filtragem casada), canal $Q$}]
{Qprime}
{\ensuremath{Q'(t)}}

% g(t) - pulso de formatação
\glsxtrnewsymbol[description={Resposta ao impulso do filtro de formatação}]
{gt}
{\ensuremath{g(t)}}

% g(-t) - pulso de formatação invertido
\glsxtrnewsymbol[description={Resposta ao impulso do filtro de formatação invertido (filtro casado)}]
{gti}
{\ensuremath{g(-t)}}

% h(t) - resposta ao impulso do filtro passa baixa
\glsxtrnewsymbol[description={Resposta ao impulso do filtro passa baixa}]
{ht}
{\ensuremath{h(t)}}

% alpha - roll-off factor
\glsxtrnewsymbol[description={Fator de roll-off do filtro RRC, $\alpha = 0.8$}]
{alpha}
{\ensuremath{\alpha}}

% W - largura de banda
\glsxtrnewsymbol[description={Largura de banda do sinal modulado, definido como $W = (1 + \alpha)/T_b$}]
{W}
{\ensuremath{W}}

% phi_0 - desvio de fase inicial
\glsxtrnewsymbol[description={Desvio de fase inicial do sinal modulado}]
{phi0}
{\ensuremath{\phi_0}}

% A - amplitude do sinal
\glsxtrnewsymbol[description={Amplitude do sinal modulado}]
{A}
{\ensuremath{A}}

% G_0 vetor gerador
\glsxtrnewsymbol[description={Vetor gerador correspondente ao, canal $I$, $G_0 = [121]_{10}$ (ARGOS-3)}]
{G0}
{\ensuremath{G_0}}

% G_1 vetor gerador
\glsxtrnewsymbol[description={Vetor gerador correspondente ao, canal $Q$, $G_1 = [91]_{10}$ (ARGOS-3)}]
{G1}
{\ensuremath{G_1}}

% componente I do sinal modulado
\glsxtrnewsymbol[description={\textit{In-phase component}}]
{cI}
{\ensuremath{I}}

% componente Q do sinal modulado
\glsxtrnewsymbol[description={\textit{Quadrature component}}]
{cQ}
{\ensuremath{Q}}

% dfree - distância livre do código
\glsxtrnewsymbol[description={Distância livre do código convolucional, $d_{free} = 10$ (ARGOS-3)}]
{dfree}
{\ensuremath{d_{free}}}

% u_t, vetor de entrada do codificador convolucional
\glsxtrnewsymbol[description={Vetor de entrada do codificador convolucional}]
{ut}
{\ensuremath{u_t}}

% v_t^{0}, vetor de saída I do codificador convolucional
\glsxtrnewsymbol[description={Saída do codificador convolucional correspondente a $G_0$}]
{vt0}
{\ensuremath{v_t^{(0)}}}

% v_t^{1}, vetor de saída Q do codificador convolucional
\glsxtrnewsymbol[description={Saída do codificador convolucional correspondente a $G_1$}]
{vt1}
{\ensuremath{v_t^{(1)}}}

% v_t^{(0) \prime}, Sequência de bits desembaralhados, canal $I$
\glsxtrnewsymbol[description={Sequência de bits desembaralhados, canal $I$}]
{vt0prime}
{\ensuremath{v_t^{(0) \prime}}}

% v_t^{(1) \prime}, Sequência de bits desembaralhados, canal $Q$
\glsxtrnewsymbol[description={Sequência de bits desembaralhados, canal $Q$}]
{vt1prime}
{\ensuremath{v_t^{(1) \prime}}}

% Xn - sequência de bits embaralhados, canal $I$
\glsxtrnewsymbol[description={Sequência de bits embaralhados, canal $I$}]
{Xn}
{\ensuremath{X[n]}}

% Yn - sequência de bits embaralhados, canal $Q$
\glsxtrnewsymbol[description={Sequência de bits embaralhados, canal $Q$}]
{Yn}
{\ensuremath{Y[n]}}

% Xn' - sequência de bits decididos após decodificação de linha, canal $I$
\glsxtrnewsymbol[description={Sequência de bits decididos após decodificação de linha, canal $I$}]
{Xnprime}
{\ensuremath{X'[n]}}

% Yn' - sequência de bits decididos após decodificação de linha, canal $Q$
\glsxtrnewsymbol[description={Sequência de bits decididos após decodificação de linha, canal $Q$}]
{Ynprime}
{\ensuremath{Y'[n]}}

% In - sequência de símbolos, canal $I$
\glsxtrnewsymbol[description={Sequência de símbolos, canal $I$}]
{In}
{\ensuremath{I[n]}} 

% Qn - sequência de símbolos, canal $Q$
\glsxtrnewsymbol[description={Sequência de símbolos, canal $Q$}]
{Qn}
{\ensuremath{Q[n]}}

% In' - sequência de símbolos decididos, canal $I$
\glsxtrnewsymbol[description={Sequência de símbolos decididos, canal $I$}]
{Inprime}
{\ensuremath{I'[n]}}

% Qn' - sequência de símbolos decididos, canal $Q$
\glsxtrnewsymbol[description={Sequência de símbolos decididos, canal $Q$}]
{Qnprime}
{\ensuremath{Q'[n]}}

% n - índice de amostra
\glsxtrnewsymbol[description={Índice de amostra/simbolo}]
{n}
{\ensuremath{n}}

% tau - delay inicial
\glsxtrnewsymbol[description={Delay inicial para decisão de símbolos}]
{tau}
{\ensuremath{\tau}}