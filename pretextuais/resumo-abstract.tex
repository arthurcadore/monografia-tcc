% ajusta o espaçamento dos parágrafos do resumo
\setlength{\absparsep}{18pt} 


\begin{resumo}
O presente trabalho tem como objetivo o desenvolvimento e simulação de um sistema de modulação e demodulação, utilizando os padrões da terceira geração do sistema de satélites ARGOS-3. Este padrão é utilizado em Plataformas de Coleta de Dados (PCDs) voltadas ao monitoramento ambiental, e emprega técnicas de comunicação digital, como modulação QPSK, codificação convolucional e embaralhamento de dados, com o objetivo de aumentar a confiabilidade do enlace de satélite. A metodologia adotada envolve o estudo de especificações do padrão de comunicação, a estrutura dos datagramas transmitidos pelas PCDs e a implementação, em ambiente simulado, dos blocos responsáveis pela transmissão e recepção digital. O conjunto de simulações vai desde a geração da portadora pura, passando pela palavra de sincronismo e codificação da mensagem do usuário, até a demodulação e recuperação dos dados transmitidos. 

    \textbf{Palavras-chave}: Comunicação por satélite; PTT-A3; ARGOS-3; Modulação digital;
\end{resumo}



%-----------------------------------------------%3
\begin{resumo}[Abstract]
\begin{otherlanguage*}{english}
The objective of this work is to develop and simulate a modulation and demodulation system based on the standards of the third generation of the ARGOS-3 satellite system. This standard is used in Data Collection Platforms (DCPs) designed for environmental monitoring and employs modern digital communication techniques such as QPSK modulation, convolutional coding, and data scrambling, aiming to increase the robustness of the satellite communication link. The adopted methodology involves a detailed study of the communication standard specifications, the structure of the datagrams transmitted by the DCPs, and the implementation, in a simulated environment, of the blocks responsible for digital transmission and reception. The proposed simulation set covers the entire chain, from the generation of the continuous wave (CW) carrier, through the synchronization word and user message encoding, to the demodulation and recovery of the transmitted data.
\vspace{\onelineskip}

\noindent 
\textbf{Keywords}: Satellite Communication; PTT-A3; ARGOS-3; Digital modulation.
\end{otherlanguage*}
\end{resumo}
%-----------------------------------------------%